\documentclass[a4paper,10pt]{article}
\usepackage[top=31mm, bottom=21mm, left=16mm, right=16mm]{geometry}
\usepackage[french]{babel}
\usepackage[utf8]{inputenc}
\usepackage{textcomp}
\usepackage{url}
\usepackage{fancyhdr}
\usepackage{color}
\usepackage[colorlinks, linkcolor=black, urlcolor=black]{hyperref}
\usepackage{graphicx}
\usepackage{listings}
\usepackage{multicol}
\usepackage{sansseriftitles}
\usepackage[scaled]{helvet}
\usepackage{engord}
\usepackage{csquotes}
\usepackage[T1]{fontenc}
\usepackage[urw-garamond]{mathdesign}

\SetBlockEnvironment{quotation}

\setlength{\columnsep}{1cm}

\hypersetup{%
  colorlinks,
  citecolor=black,
  citebordercolor=1 0 0,
  linkcolor=black,
  urlcolor=blue,
  pdfborder=true
}

\pagestyle{fancy}
\lhead{Paul \textsc{ADENOT} \texttt{<paul@paul.cx>}}
\rhead{Département informatique, année 2011/2012}
\cfoot{\thepage}
\setlength{\parskip}{1em}
\setlength{\headheight}{13.6pt}

\newcommand{\cc}[1]{\texttt{#1}}
\title{\textsf{\textbf{Rapport de Synthèse\\Implémentation de nouvelles
fonctionnalités media dans Gecko}}}
\author{Paul \textsc{Adenot}, \texttt{<paul@paul.cx>}}
\date{}

\begin{document}
\maketitle
\begin{multicols}{2}
\subsection*{Entreprise d'accueil:}
\noindent
Mozilla Corporation\\
650 Castro Street\\
Suite 300\\
Mountain View, CA, 94041-2021\\
USA\\

\subsection*{Enseignant responsable:}
\noindent
Előd \textsc{Egyed-Zsigmond}

\section*{Résumé}


\section*{Mots-clefs}

\section*{Abstract}
This paper synthesizes the state of the field of medical informatics, through
the analyse of four papers. Main problems and challenge encountered when
developing IT systems for medical professionals are presented, as well as
solutions researchers have found. A parallel is drawn with the sibling field of
Information System development, and with Agile software development. Several
characteristics of successful projects are isolated, and are debated along with
known best practices of the software industry.

\section*{Keywords}

\columnbreak

{\footnotesize
\tableofcontents
}

\clearpage

\section{Introduction}

Human-computer interaction takes nowadays an important part of our lives.
However, in the workplace, while computers are omnipresent, numerous are the
uses of technology which could be improved. By trying to modernise the way
people
work using technology, professionals are facing numerous problems.
This paper will focus particularly on the healthcare domain, and will look,
through four %XXX
papers, at the state of the technology in hospitals, from different points of
view: IT professionals, healthcare professionals, and HCI researchers. While
there is interesting technical challenges to solve, we will particularly focus
on the \emph{social} aspect of the problems: the relations between users and
professionals, the impact of the technology on the care given by the healthcare
professionals, and related concerns.

\section{Main challenges}

The field of Human-Computer interaction is facing numerous challenges in the
domain of medical informatics.
While other domains have been fully equipped in IT solutions for years, it seems
that the healthcare industry is lagging behind.

At first sight, the methodological findings in this field seem to be
\emph{rediscoveries}, reading the papers studied in the different seminars.
Ideas are often not far from the agile management basic principles, which was
formalised in 2001\cite{AgileManifesto} for the software domain, but was
informally used (or at least partly) several years before that.

These main challenges are:
\begin{itemize}
  \item The current level of acceptance rate by the professionals of the new
  system proposition is too low. This results either in waste of money if the % XXX
  solution is not used, or a waste of time, and users frustrations, if the
  solution is used.
  \item The productivity needs to be increased, as well as the level of the care,
  in term of quality. The technical solution should not be an obstacle to giving % XXX
  of care.
  \item The IT industry often wants to push technology in the hospitals, more % XXX
  than they want people to be taken care of. % o rly ?
  \item In an hospital, bureaucracy and politics are very hard to work with.
  Often, management and medical staff have different goals in mind when it    %XXX
  comes to decide of a new IT system.
  \item Medical staff time is very valuable, it is difficult to justify the use
  of their time to test a new system or conducting interviews, while they could %XXX
  be giving care to needy people.
  \item Reliability is important, failure (or even slowdown) are not acceptable
  for many applications. The healthcare domain wants proofs before any action is
  taken. The current research are still trying to find factual proofs for       %XXX
  methods. \emph{Best practices} used in other domains may not be applicable
  in the healthcare industry.
  \item The systems should be able to cope with the heterogeneous level when it
  comes to working with computers of the medical staff. Most of the               %XXX
  professionals are not trained for advanced computer usage (this point beginning
  to get gradually better, thanks to the introduction of new technologies at
  home, as stated by \cite{IndiaEMR} and \cite{UsersAward2}).
  \item The system should be able to scale among the entire organisation.
\end{itemize}

In the next sections, attempts to answer these challenges will be discussed.
\section{Problems area}

\subsection{Users}
At first sight, it seems natural in the healthcare domain, to use a user-centric
design. However, this rather old and proven way of building a solution didn't
appear immediately to the researchers. Indeed, while in other domain, solution
design that incorporate the end-user very early in the process is often used, it
hasn't been used in the first deployment related in \cite{IndiaEMR}, and
that led to a total rejection of the system by the end users.

Agile method users claim that their method responds to this kind of problems.
The manifesto \cite{AgileManifesto} states, among other principles:

\begin{quote}
«~\emph{Our highest priority is to satisfy the customer through early and
continuous delivery of valuable software.}~»
\end{quote}

\begin{quote}
«~\emph{Business people and developers must work together daily throughout the
project.}~»
\end{quote}

\emph{Business people} here should be understood as people from the business
where the solution is going to be deployed in, medical staff in our case.

However, often, medical staff cannot be part of the whole experiment, the
workload on a day-to-day basis does not allow them to work even more, so they
must somehow cut on the time they allocate to the care, and participate to these
experiments. The concept of an \emph{agile champion} can help answer to this
concern. One person volunteers to be the representative of the team (let's say, a
nurse among all the nurses), and is involved in the whole process of change,
giving insights, gives feedback, and bridges the medical staff and the people
who are building the system. While the term \emph{agile champion} is not used,
this is certainly the concept that is used in \cite{IndiaEMR}, bringing
acceptance where the previous method failed, possibly because of lack of
communication, or a lack of knowledge of the business rules.

However, \cite{Hasvold2011} shows an example where user involvement lead to
failure, which shows that user insight and suggestions should be taken with
precaution. The nurses have asked for a system, which was built for them, and
promptly rejected afterwards. That shows that blindly following user demand is
not a good solution either. Moreover, often, people don't know what they want,
and don't come up with innovative solutions. A person whose job is to build
system for hospitals is more likely to have pioneering ideas than a doctor, as
stated by this famous quote, by Henry Ford:

\begin{quote}
  «~\emph{If I'd asked customers what they wanted, they would have said ``a
  faster horse''.}~»
\end{quote}

We have seen in \cite{IndiaEMR} and \cite{Jalote2010} that the
innovation came from the technology industry, which tried to use new products in
the workplace in order to try to improve productivity. While it can be bad to
push technology in order to drive the innovation in the health-care domain when
it is not needed, pushing medical professionals a little, to show them new
products, new ideas, new ways or working, and see reaction has shown to be
beneficial. Even in the case of failure, like in \cite{Hasvold2011}, the use
of agile methods enables the project to pivot rapidly to a new system (which is
the flat-screen display in the nurse HQ), which gets general approbation.

This brings us to the question of the amount of \emph{trust} we put in the users of
the system. Taking a step back from all these methods and considerations, it can
be concluded that as scientists, teams should not \emph{trust} users. They
should build solutions upon quantitative and comprehensive data, gathered by
field studies or other means. However, sometime, the only way to gather data,
and to see what works and what doesn't is to try something and to fail, as presented in
\cite{Hasvold2011}.

Finally, we see that a user-centric design using an agile approach can be
applied (and has been applied) with success several times, leveraging the work
done in the software industry. However, the design, here, is not really
\emph{user}-centric. The ultimate goal, being improving the quality of the care
given by the professionals, should always be remembered. The method employed is
therefore more of a \emph{patient-centric} design than a \emph{user-centric}
design.

\subsection{Youth and particularities of the domain}
The domain of medical informatics is quite young, and, where one could think
that it can leverage the decades of studies on methodologies done in the software development
industry, the requirements needed by hospitals make difficult
taking all the knowledge available for granted. Hence, there is a need for some
\emph{rediscoveries}, which could pass as the researchers where reinventing the
wheel. However, we see in \cite{Hasvold2011} that the team uses the Scrum method
successfully, so some can of reuse can be performed.

Moreover, in order to make changes in the organisation of an hospital (which is
likely to happen, see~\ref{mapping}), hierarchy in hospital often being rigid,
it is necessary to have factual proofs that the changes are going to improve
a process. Momentum must be accumulated, little by little, by research papers,
which are indeed factual proof, given they contain quantitative
data, which is not always the case (\cite{Hasvold2011} does not contain
quantitative data, for example, but only subjective \emph{perception} of what is
going on in the organisation).
This domain is also voluntarily considered by the researchers as a domain on its
own, separated from other fields.

\subsection{Scalability}

The \emph{scalability}, in the domain of medical informatics has two meanings.

It can refer to the usual concept, found in other computer-related domains,
refers to the notion that a system could have an acceptable behavior under a
certain load, but as the load (the number of users using the product) increase,
the speed becomes problematic.  This becomes particularly a problem, since
delays, in emergency situation where all the staff is likely to interact with
the system, could not be acceptable. This technical meaning will not be explored
in this paper.

Along with this first meaning, a system \emph{scales} well if the complexity doesn't
explode when the system expands out of the pilot service where it is likely that
it will be deployed first. According to \emph{UsersAward2}, the solution will be
found by «~…\emph{model[-ing] the domain and its sub-domains more closely~}».
Domain modeling, that is, modeling the business relation between actors, can be
of use to find patterns on a huge system. By a careful systematic approach, one
can more precisely understand how the organization works, and provide answers
that take into consideration the characteristic of a group of users, in order to
build a solution that fit this particular group of users.

\section{Solutions area}

\subsection{Development models encountered}

The papers show two different types of development scheme, that both seem to
work, in the context of the creation of an IT system for an hospital:
\begin{itemize}
\item Iterative development.
\item Linear development.
\end{itemize}

The iterative development follows the \emph{agile} way of writing software:
short, successive, iteration, without fear of throwing away ideas (or selecting only
the ideas that are realistic, \cite{Hasvold2011}), lightweight
process, maximum user implication. If the features implemented don't work, the
method ensures that it is possible to pivot to a new solution using updated
requirements (\emph{via} user insight, interviews, and other various feedbacks
means). This method has been applied in the paper \cite{Hasvold2011}, where the
team, after a first failure, changes its direction, and succeeds. This method
can be summarized as the quote, from Eric S. Raymond, in \cite{Raymond97} :
«~Release early, release often. And listen to you customers.~»

The linear development implements the system from top to bottom, perhaps having
a future user (the champion) in-house to maximise the chances of the acceptance
of the solution. A lot of information is gathered beforehand, and acceptance
testing is done in the latest part of the project. This type of workflow is easier to
outsource, and is closer to the old
\emph{waterfall}\footnote{\url{http://en.wikipedia.org/wiki/Waterfall\_model}} model.

While the first method is dubbed more \emph{modern}, it demands more involvement
by the staff, which is not easy to obtain, as seen before.

\subsection{Functional mapping, users alignment}
\label{mapping}

\emph{Functional mapping} and \emph{user alignment} are two core concepts in the
ERP (Enterprise Resource Planning) domain, and allow the reuse of solutions,
based on \emph{best practices}\footnote{ISO has standardised plenty of
behavioral practices for medical informatics:
\url{http://www.iso.org/iso/iso_catalogue/catalogue_ics/catalogue_ics_browse.htm?ICS1=35&ICS2=240&ICS3=80&published=on&development=on&withdrawn=on}},
which describes medical informatics system from the security point of view. When
implementing such a standard, vendors can come up with ready-to-install
solutions, which saves time and money. The functionalities of the software are
mapped to the roles of the organisation.  After that part, functional gaps can
subsist, hence the so-called \emph{users-alignment} phase, which fills this gap
by changing the way the organisation works. If is is impossible to change the
workflow of the organisation, for some reason, \emph{specific development} will
have to be performed to bring the solution at its full potential.

\subsection{Information gathering}

All the papers relate the use participatory design. Even if it can seem to fail
sometimes (for example in \cite{Hasvold2011}), it appears that it has been the
solution. A quote from \cite{UsersAward2} is categorical, and is based on factual
data:

\begin{quote}
«~[…] in the new report we conclude that, in order to increase levels of
satisfaction, more participatory-design-for-innovation R\&D programmes have to be
carried out focusing on how innovative local initiatives can be adapted,
replicated and spread between caregivers in a way that conforms with regional
and national policies.~»
\end{quote}

\section{Equality and diversity}

The installation of a computer system in an hospital must take into account the
likeliness of the heterogeneity of the users of the system. If a young medical
secretary should be able to handle the change and a state-of-the-art
information system, other people, especially when they are at the end of their
careers, will have a hard time grasping the way the system works, and will
reach a high efficiency only at the cost of a long training.

In such systems, relying on the so-called \emph{snowball effect} can be an
effective way to fill the gap between users of the system, and users that are
not using it because of the lack of technical knowledge.

In \cite{IndiaEMR}, which takes place in India, it is mentioned the fact
that the staff hasn't got an access to computers at home, which limits their
% imitates ??
learning. The hospital takes the initiative of leveraging the knowledge of its
IT staff to facilitate the initial setup and the maintenance of computers at
home, for the medical staff to get familiar with the technology.

In the same paper, another solution has been tried, with success. The GUI
(Graphical User Interface) of the programs that were installed as part of the
solution were designed to be \emph{as similar as possible} to the existing paper
records that were used in this hospital.

If the idea of having both the legacy paper-and-pen and the new computer-based
system is feasible, it could be a great advantage, making the transition
smoother (especially if the idea presented in the previous paragraph has been
implemented). This has to be considered with care, since it is likely that it
will bring a certain overhead: the data created using the old system will have
to be transfered into the new one.

A study, conducted both in 2004 and 2010 \cite{UsersAward2}, shows that the
experience in ICT among the different kind of medical profession found in an
hospital is getting more uniform. The overall bad result in this study is
explained by the authors by the rise of the expectations among the professionals.

%\section{Conclusions}
%Despite facing numerous hard problem and being a relatively new field, the
%medical informatics domain has been able to give itself analysis tools and
%leverage the work done in other field, while being sure that it could be
%applicable, by field studies and other research means. 

\bibliographystyle{abstract}
\bibliography{PaulADENOT-Report.bib}

\end{multicols}

\end{document}

