\documentclass[a4paper,10pt]{article}
\usepackage[top=31mm, bottom=21mm, left=16mm, right=16mm]{geometry}
\usepackage[french]{babel}
\usepackage[utf8]{inputenc}
\usepackage{textcomp}
\usepackage{url}
\usepackage{fancyhdr}
\usepackage{color}
\usepackage[colorlinks, linkcolor=black, urlcolor=black]{hyperref}
\usepackage{graphicx}
\usepackage{listings}
\usepackage{multicol}
\usepackage{sansseriftitles}
\usepackage[scaled]{helvet}
\usepackage{engord}
\usepackage{csquotes}
\usepackage[T1]{fontenc}
\usepackage[urw-garamond]{mathdesign}

\SetBlockEnvironment{quotation}

\setlength{\columnsep}{1cm}

\hypersetup{%
  colorlinks,
  citecolor=black,
  citebordercolor=1 0 0,
  linkcolor=black,
  urlcolor=blue,
  pdfborder=true
}

\pagestyle{fancy}
\lhead{Paul \textsc{ADENOT} \texttt{<paul@paul.cx>}}
\rhead{Département informatique, année 2011/2012}
\cfoot{\thepage}
\setlength{\parskip}{1em}
\setlength{\headheight}{13.6pt}

\newcommand{\cc}[1]{\texttt{#1}}
\title{\textsf{\textbf{Rapport de Synthèse\\Implémentation de nouvelles
fonctionnalités media dans Gecko}}}
\author{Paul \textsc{Adenot}, \texttt{<paul@paul.cx>}}
\date{}

\begin{document}
\maketitle
\subsection*{Entreprise d'accueil:}
\noindent
Mozilla Corporation\\
650 Castro Street\\
Suite 300\\
Mountain View, CA, 94041-2021\\
USA\\

\subsection*{Enseignant responsable:}
\noindent
Előd \textsc{Egyed-Zsigmond}

\section*{Résumé}
La fondation Mozilla est une organisation à but non lucratif qui a pour but de
promouvoir un Web ouvert à tous, par l'intermédiaire de son produit phare,
le navigateur web Firefox. Durant ce stage, de nouvelles fonctionnalités dans
le composant media du logiciel ont été implémentés, ainsi que des optimisation
de la vitesse de rendu des documents. Il sera aussi évoqué les processus
particuliers de travail que l'entreprise a adopté, ainsi que ses relation avec
les organismes de standardisation des normes qui font le web, de nos jour.

\section*{Mots-clefs}
Mozilla, logiciel libre, media, thread, codec, travail distribué,
réechantillonage, time-stretching, type MIME, media query, box blur, dégradé,
spécification, Web.
\section*{Abstract}
The Mozilla Foundation is a non-profit organization whose goal is to promote a
open Web, by creating and maintaining its main product, the Firefox web browser.
During this internship, new features of the media component of Firefox have been
implemented, along with a few optimizations of the rendering speed of documents.
There will be mention of the particular processes the organization had to adopt,
and its relation to the organizations that write the specifications for the web
technologies.

\section*{Keywords}
Mozilla, free software, media, thread, codec, remote working, resampling,
time-stretching, MIME type, media query, box blur, gradient, specification, web.

{\footnotesize
\tableofcontents
}

\clearpage

\begin{multicols}{2}
  \part*{Introduction}
  \part{Contexte}
  \section{Vue d'ensemble de Firefox}
  \section{Vue d'ensemble du sous-sytème media}
  \part{Sythèse technique}
  \section{Proprieté \cc{playbackRate} des éléments \cc{<audio>} et \cc{<video>}}
  \section{Attribut \cc{media} de la balise \cc{<source>}}
  \section{Détection de type MIME par reniflage}
    \subsection{Risque de sécurité}
    \subsection{Implémentation}
  \section{Mise en cache du dessin des dégradés}
  \section{Optimisation des routines de flou pour dessiner les ombres}
  \part{Synthèse au niveau processus}
  \section{Vie d'une fonctionnalité dans Gecko}
  \section{Environnement de travail}
  \part{Bilan}

%\bibliographystyle{abstract}
%\bibliography{PaulADENOT-Report.bib}

\end{multicols}

\end{document}

